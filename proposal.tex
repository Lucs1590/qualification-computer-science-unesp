\newpage
\clearpage
\section{Proposta de Segmentação Panóptica em Componentes Visuais}
\label{proposal:proposal}


\subsection{Materiais e Métodos}
\label{proposal:matmet}
Nesta Seção, serão discorridos detalhes sobre as tecnologias que serão utilizadas para o desenvolvimento de um modelo de segmentação panóptica que segmenta componentes visuais (Subseção \ref{proposal:tec}), sobre os conjuntos de dados e as estratégias utilizadas para cumprirem com o objetivo da segmentação (Subseção \ref{proposal:dataset}) e, por fim, as técnicas e métodos escolhidos para serem utilizados nos experimentos (Subseção \ref{proposal:method}).


\subsubsection{Tecnologias}
\label{proposal:tec}

Dentre as tecnologias planejadas para o desenvolvimento do presente trabalho, vale citar que destaca-se o uso da linguagem de programação interpretada e de alto nível, \textit{Python}, sendo essa uma linguagem vantajosa por ser popularmente conhecida em meio cientifico, com uma ampla comunidade e com  bibliotecas dispostas para facilitar o desenvolvimento de soluções \cite{Millman2011PythonEngineers}.

Já em relação às bibliotecas planejadas ao projeto, destacam-se aquelas cujo desenvolvimento é amplamente utilizado para o auxilio de projetos de visão computacional, aprendizado de máquina e redes neurais, das quais cita-se: \textit{PyTorch}, \textit{Keras}, \textit{TensorFlow}, \textit{OpenCV}, \textit{Numpy}, \textit{Pandas}, entre outras.

Por fim, em relação aos métodos desenvolvido para cumprir com os objetivos do presente trabalho, no que lhes concerne, de realizar a segmentação de componentes visuais, declara-se que serão disponibilizado de modo \textit{Open Source} por meio do \textit{Github} do autor\footnote{Perfil \textit{Github} do autor – \url{https://github.com/Lucs1590}}, segundo os preceitos a licença Apache v2.0 \cite{Licenses}, com o intuito de contribuir com o crescimento de futuros pesquisadores, além de possibilitar futuras melhorias da aréa de segmentação com uso de segmentação hierárquica.


\subsubsection{Conjuntos de Dados}
\label{proposal:dataset}

Em relação ao conjuto de dados, ressalta-se que ...

\subsubsection{Método}
\label{proposal:method}

\subsection{Metodologia}
\label{proposal:methodology}

\subsubsection{Transferência de Aprendizado}
\label{proposal:transf}

\subsubsection{Avaliação}
\label{proposal:avaliation}

\subsection{Revisão de Literatura}
\label{proposal:revision}

\subsection{Cronograma}
\label{proposal:cron}
